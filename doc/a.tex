\documentclass[a4paper]{article}

\title{Simplest Blob Detection}
\author{EML \& RGvG}

\begin{document}

\maketitle
\begin{abstract}
	We present a simple and fast multi-scale blob detector.
	The ``blobs'' are defined as well-contrasted local maxima or minima of
	the image with a round shape.
	It is a straightforward differential criterion based on the Hessian
	matrix of the image filtered at different scales, requiring about 18
	multiplications per pixel.
	The criteria for the proposed implementation are, in decreasing order
	of importance: (1) simplicity of the algorithm, (2) minimal
	computational cost, (3) quality of the result.
	We believe that our implementation is the shortest possible code that
	is able to detect white and black blobs of any size in images.
\end{abstract}

\section{Introduction}

short history of feature detection: edges, points, etc

keypoint detection: corners vs. local extrema vs. centers of blobs

\section{Simplest scale-space}

poor man gaussian filtering

poor man downsampling

construction of the image pyramid

\section{Simplest blob detection at one scale}

criterion in the continuous domain:

1) local minima of the $\sigma$-filtered image

2) such that $u_{xx}+u_{yy} > \tau$ and
$\frac{u_{xx}u_{yy}-u_{xy}^2}{(u_{xx}+u_{yy})^2}>\kappa$

discretization

poor man gaussian filtering

second derivatives by finite differences at the closest octave


\section{Simplest sub-pixel and sub-octave localization}

separable sub-pixel localisation by fitting a parabola on each direction and
finding its minimum

\section{Experiments and conclusion}

show some results, give some numbers as FPS using a few combinations of
architecture/video resolution/compiler+options

\end{document}

% vim:set tw=79 spell spelllang=en:
