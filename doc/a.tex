\documentclass[a4paper]{article}
\usepackage{amsmath}
\usepackage[vlined,algoruled,linesnumbered]{algorithm2e}
\usepackage{xcolor}
\def\R{\mathbf{R}}


\title{Harressian: a Simple Blob Detector}
\author{EML \& RGvG}


\begin{document}

\maketitle
\begin{abstract}
	We present a simple and fast multi-scale blob detector.
	The ``blobs'' are defined as well-contrasted local maxima or minima of
	the image with a round shape.
	The method is a straightforward differential criterion based on the
	Hessian matrix of the image filtered at different scales, requiring
	about 18 multiplications per pixel.
	The criteria for the proposed implementation are, in decreasing order
	of importance: (1) simplicity of the algorithm, (2) minimal
	computational cost, (3) quality of the result.
	Our implementation is among the shortest possible codes that
	are able to detect white and black blobs of any size in images.
\end{abstract}

\section{Introduction}

Blob detection is the task of finding round and well-contrasted objects in an
image.  See Figure~\ref{fig:blobsandnot} for examples of blobs and non-blobs.
Notice that are two kinds of blobs: dark blobs on a light background and light
blobs on a dark background.  In this article we focus only on the first kind
(dark blobs); the case of light blobs is be obtained by applying the same
algorithm to the negative image.

one paragrah with short history of feature detection: edges, points, etc

one paragraph about keypoint detection: corners vs. local extrema vs. centers
of blobs

\section{Overview}

Let us model an image as a smooth function~$u:\R^2\to\R$.
We use the following four criteria for detecting (dark) blobs:
\begin{enumerate}
	\item[B1] The position of a blob is a local minimum of the image~$u$
	\item[B2] These center of a blob is much darker than the neighboring points
	\item[B3] The level lines of $u$ around a blob are approximately circular
	\item[B4] Blobs may occur at any scale
\end{enumerate}
For digital images, the first criterion B1 is trivial to check: look at the
8 neighboring pixels, and if any of them is darker, the central pixel is not a
blob.  This criterion discards immediately around
$87.5\%=100\left(1-\frac{1}{8}\right)$ of the image domain.

To formalize criteria B2 and B3, we look at the second derivatives of~$u$
(since we already know that we are on a minimum, the first derivatives cannot
give us any more information).  The second derivatives are arranged in the
Hessian matrix:
\[
	H(u)=\begin{pmatrix}u_{xx} & u_{xy} \\ u_{xy} & u_{yy}\end{pmatrix}
\]
we extract two descriptors based on the trace and the determinant of this
matrix:
\[
	\mathrm{deepness}(u)=u_{xx}+u_{yy}
\]
\[
	\mathrm{roundness}(u)=\frac{u_{xx}u_{yy} - u_{xy}^2}{\left(u_{xx}+u_{yy} \right)^2}
\]
The deepness is the Laplacian of the image.  On local minima, the Laplacian is
always positive, and its value is larger when the minimum is deeper.

The roundness is the normalized ratio of the determinant and the trace
of~$H(u)$, and it is a non-dimensional quantity with values in the
interval~$[0,\frac{1}{4}]$.
The name ``roundness'' is justified by observing for which functions~$u$ are
these extremal values attained.  When the graph of~$u(x,y)=x^2+y^2$ is a
paraboloid of revolution, whose level lines are circles, the roundness at the
minimum~$(0,0)$ is~$\frac{1}{4}$.  When~$u(x,y)=x^2$ or~$u(x,y)=(x+y)^2$, the
graph of~$u$ is a parabolic cylinder, the level lines are a collection of
parallel straight lines, and the roundness at the minimum is exactly~$0$.  An
intermediate case is for example~$u(x,y)=x^2+\alpha y^2$, for~$0 < \alpha < 1$,
the level lines are ellipses and the value of the roundness is monotonic with
their eccentricity.

The criterion B2 can be formalized as~$\mathrm{deepness}(u) > \tau$,
where~$\tau$ is a threshold.  Since we do not want to find any blobs on flat
regions of the image, a good default value for~$\tau$ is proportional to the
expected level of noise in the image.

The criterion B3 can be formalized as $\mathrm{roundness}(u) > \kappa$,
where~$\kappa$ is a threshold.  Since we want blobs that are as round as
possible we set~$\kappa$ to a value which is very close but smaller
than~$\frac{1}{4}$, for example~$\kappa=0.24$.

The criterion B4 {\color{red} rafa say something intelligent about multi-scale}.

The high-level pseudo-code of the algorithm is thus described
on~Algorithm~\ref{alg:hilevel}.

\begin{algorithm}
	\caption{\texttt{harressian}}
	\label{alg:hilevel}
	\DontPrintSemicolon
	\SetKwInOut{Input}{Input}
	\SetKwInOut{Output}{Output}
	\SetKwFunction{multiscalepyramid}{multi-scale-pyramid}
	\SetKwFunction{harressiannogauss}{harressian-single-scale}
	\SetKwFunction{laplacian}{laplacian}
	\Input{Image~$I$}
	\Output{List of blob locations~$L=\{(x,y,\sigma)\}$}
	\BlankLine
	$L\gets\emptyset$\;
	$(I_0,\ldots,I_n)\gets\multiscalepyramid(I)$\;
	\For{$i=0,\ldots,n$}{
		$L\gets L\,\cup\,\harressiannogauss(I_i)$\;
		}
	\For{$(x,y,\sigma)\in L$}{
		$i\gets\left\lfloor\log_2(\sigma)\right\rfloor$\;
		$a\gets\laplacian(I_{i+1},x,y)$\;
		$b\gets\laplacian(I_{i\phantom{+1}},x,y)$\;
		$c\gets\laplacian(I_{i-1},x,y)$\;
		\If{$a>b$ or $c>b$}{
			$L\gets L\setminus\{(x,y,\sigma)\}$\;
			}
		}
	\Return{$L$}\;
\end{algorithm}


\section{Blob detection at one scale}

Criterion in the continuous domain:

1) local minima of the $\sigma$-filtered image

2) such that $u_{xx}+u_{yy} > \tau$ and
$\frac{u_{xx}u_{yy}-u_{xy}^2}{(u_{xx}+u_{yy})^2}>\kappa$

discretization

Poor man gaussian filtering

Second derivatives by finite differences at the closest octave


\section{Scale-space}

Poor man gaussian filtering

Poor man downsampling

Construction of the image pyramid

The basic data structure for exploring such large images is the multi-scale
pyramid. This data structure is simple: it is the union of several
vertically down-scaled versions of the original image, by factors that are
successive powers of two, called ``octaves''.  The construction of this
data structure is detailed in algorithm~\ref{alg:pyramid}.

\begin{algorithm}
	\caption{\texttt{build-multiscale-pyramid}}
	\label{alg:pyramid}
	\DontPrintSemicolon
	\SetKwInOut{Input}{Input}
	\SetKwInOut{Output}{Output}
	\SetKwFunction{zoomout}{zoom-out}
	\SetKwFunction{zoomin}{zoom-in}
	\SetKwFunction{approximategaussianblur}{approximate-gaussian-blur}
	\Input{Image~$I$ of size $W\times H$}
	\Output{List of images~$I_0,\ldots,I_N$}
	\BlankLine
	$\sigma\gets 1.4$\;
	$N\gets \left\lfloor\log_2\min(W,H)\right\rfloor$\;
	$I_0\leftarrow I$\;
	\For{$n=1,\ldots,N$}{
		$J\gets\approximategaussianblur(I_{n-1},\sigma)$\;
		$I_n\leftarrow\ \textrm{new image of size~$
		\left\lceil\frac{W}{2^n}\right\rceil
		\times
		\left\lceil\frac{H}{2^n}\right\rceil
		$}$\;
		\For{$(i,j)\in I_n$}{
			$I_n(i,j)\leftarrow J(2i,2j)$\;
		}
	}
\end{algorithm}

\section{Simplest sub-pixel and sub-octave localization}

separable sub-pixel localisation by fitting a parabola on each direction and
finding its minimum (devernay localisation)

\section{Experiments and conclusion}

show some results, give some numbers as FPS using a few combinations of
architecture/video resolution/compiler+options

\end{document}

% vim:set tw=79 spell spelllang=en:
